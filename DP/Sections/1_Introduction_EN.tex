\section{Introduction}\label{sec:DP}
\begin{tcolorbox}[title=Data Validation]
\textbf{Martin K:} Data is messy, Alison.  \\ 
\textbf{Alison M:} Even when it's been cleaned?  \\ 
\textbf{Martin K:} Especially when it's been cleaned.\\[-0.6cm]
\begin{flushright}
-- P. Boily, J. Schellinck, \textit{The Great Balancing Act}.
\end{flushright}
\end{tcolorbox}
\noindent
Data cleaning and data processing are essential aspects of quantitative analysis projects; analysts and consultants should be prepared to spend up to 80\% of their time on data preparation,  keeping in mind that:
\begin{itemize}[noitemsep]
\item processing should \textbf{NEVER} be done on the original dataset -- make copies along the way;
\item {\textbf{ALL}} cleaning steps need to be documented;
\item if \textbf{too much} of the data requires cleaning up, the data collection procedure might need to be \textbf{revisited}, and 
\item records should only be discarded as a \textbf{last resort}.
\end{itemize}
Another thing to keep in mind is that cleaning and processing may need to take place more than once depending on the type of data collection (one pass, batch, continuously), and that that it is essentially impossible to determine if all data issues have been found and fixed.\footnote{In this report, we are assuming that the datasets of interest contain only numerical and/or categorical observations. Additional steps must be taken when dealing with unstructured data, such as text or images.}
\newpage

