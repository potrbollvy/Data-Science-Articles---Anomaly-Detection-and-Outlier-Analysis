%----------------------------------------------------
%	ABSTRACT
%----------------------------------------------------

\Abstract{ 
La collecte automatique de donn\'ees permet d'emmagasiner des ensembles de donn\'es massifs. Outre les d\'efis li\'es \`a la vitesse et l'efficacit\'e des analyses de telles donn\'ees, la détection des anomalies et l’analyse des valeurs aberrantes reçoivent \'egalement une attention particulière. \par\bigskip Ces valeurs extrêmes et irrégulières ont des comportements très différents de la majorité des observations. Elles peuvent repr\'esenter, par exemple, des attaques criminelles, des tentatives de fraude, des attaques ciblées, ou encore des erreurs de collecte.  La détection des anomalies et l’analyse des valeurs aberrantes joue donc un rôle fondamental en cybersécurité, en contrôle de la qualité, et ainsi de suite \cite{A1,A2,A3}. Les lourdes cons\'equences que l'existence de ces donn\'ees peut engendrer (non-seulement du c\^ot\'e technique, mais \'egalement dans un volet humain) en font un d\'efi \`a relever \`a bras-le-corps. \par\bigskip
Dans ce rapport, nous pr\'esentons une revue de diverses méthodes de d\'etection et nous effectuons une analyse comparative de ces m\'ethodes dans le but d’évaluer leurs performances et leurs limitations à travers quelques exemples pratiques. Nous portons plus particuli\`erement notre attention \`a la détection supervis\'ee et \`a la d\'etection non-supervis\'ee.\par\smallskip \ 
}
%----------------------------------------------------