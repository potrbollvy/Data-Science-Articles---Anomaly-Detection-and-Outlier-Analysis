%----------------------------------------------------
%	ABSTRACT
%----------------------------------------------------

\Abstract{ 
Depuis quelque années, la détection des anomalies et l’analyse des valeurs aberrantes reçoivent une attention particulière des analystes de données. En effet, les valeurs aberrantes (anomalies)  sont  des valeurs extrêmes  dans les données  qui ont des comportements très différents de la  majorité. Par example, elle peut representer:  les attaques criminelles, des  fraudes, des attaques ciblées via internet dans le but de voler des  informations secrètes à d’autres fins.  On trouve la détection des anomalies et l’analyse des valeurs aberrantes dans beaucoup de domaines :  la cybersécurité, le système de detection d’intrusions, la detection des transactions frauduleuses: fraude de carte de crédit, le contrôle de qualité etc. \cite{A1}.  Plusieurs auteurs ont abordes ce sujet du fait de son importance et ses innombrables applications (\cite{A1}, \cite{A2}, \cite{A3} ). Cependant, il reste un défis à prendre à bras-le corps du fait des consequences lourdes qu’elles peuvent engendrées. \newline

Dans ce project, nous avons étayé une revue des différentes méthodes  puis nous avons  déduit une analyse comparative dans le but d’évaluer leurs performances ainsi que leurs limitations à travers des exemples. Une attention particulière  a été sur la détection des anomalies dans le cas supervisé et non supervisé. Des exemples pratiques ont fait  l’objet d’une étude détaillée avec des données réelles et synthétiques.
}

%----------------------------------------------------