\section{Détection des valeurs aberrantes dans les données qualitatives \\ 
 \textcolor{red}{Soufiane} }
%
\subsection{Definition and défis}
%

\subsubsection{Variables catégoriques}
Une variable catégorique, appartenant à l’échelle nominale  ( Des exemples de ce type sont la couleur, la direction, la langue, etc.). 
La tendance centrale des valeurs d'une variable catégorique est donnée par son mode, alors que des valeurs numériques puissent apparaître correspondant à une variable catégorique, elles représentent chacune un concept distinct et ne peuvent pas être traitées comme des nombres. 
Une variable catégorielle pouvant prendre exactement deux valeurs est appelée variable dichotomique (binaire). Les variables avec plus de deux valeurs possibles sont appelées variables polytomiques. L'analyse de régression sur les variables catégorielles est réalisée par régression logistique multinomiale.



\subsubsection{Défis avec les données catégoriques}
Lorsqu'on traite des données ayant des attributs catégoriels, on suppose que ces attributs  pourraient facilement être representer en valeurs numériques. Cependant, il existe des exemples d'attributs catégoriels, dans lesquels ces represenration en attributs numériques n'est pas un processus simple. Par conséquent, des méthodes telles que celles basées sur des mesures de distance ou de densité ne sont pas acceptables, nécessitant une modification nécessaire de leurs formulations.

\subsection{Revue de quelques méthodes}

\subsubsection{Algorithm Greedy}



\subsubsection{Algorithm AVF}


