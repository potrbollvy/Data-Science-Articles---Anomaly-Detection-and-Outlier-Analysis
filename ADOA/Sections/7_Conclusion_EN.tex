\section{Conclusion}
 Anomaly detection and outlier analysis is a vast and interesting field. The first part provides a structured overview of the different methods in the quantitative and qualitative case, in small as well as in large dimensions. Then, special attention is paid to HDLSS (High dimension low sample size), as it remains a very open area with many challenges to be addressed: the so called \textbf{curse of dimensionality}, for example. Thus, the majority of the methods presented in Section \ref{Section:2} can no longer be applied.  Through an extensive study on a series of real life dataset in unsupervised and supervised approaches of some classical methods mentioned above, allowed to understand of  the strengths and limitations of these methods. It is therefore, important to know that there is no perfect method for all datasets, in fact they are data and domain dependant. However, some are useful . In fact, \textbf{George Box} said:\textbf{<< All models are wrong but some are useful >>}. The practical question is \textbf{<< how wrong do they have to be to not be useful>>}.

\afterpage{\FloatBarrier}
