
\section{Conclusion}
La detection des anomalies et l'analyse  des valeurs aberrantes est un domaine vaste et intéressant. La prémière partie fournit un aperçu structuré des différentes méthodes dans le cas quantitatif et qualitatif, en petite aussi bien qu'en grande dimension. Ensuite, une attention particulière est pourtée sur le HDLSS (High dimension low
sample size), car il reste un domaine très ouvert avec de nombreux défis à relever: le fléau de la dimensionnalité, par exemple. Ainsi, la majorité des méthodes présentées dans la section \ref{Section:2} ne peuvent plus être appliquées. 
Enfin, a travers une étude extensive sur une série de données réelles des différentes approches non-supervisées  et supervisées de quelques méthodes classiques  mentionnées ci-haut, a permis de comprendre les atouts et limites ces méthodes.  Il est  donc important de savoir qu'il n'existe pas de méthode parfaite dominante pour tous les ensembles de données, en fait, elles dépendent plutôt des données et du domaine d'application. Cependant, certaines sont utiles. \textbf{George Box} a dit: \textbf{"Tous les modèles sont faux mais certains sont utiles”}. La question à poser est :\textbf{"A quel point doivent-ils être mauvais pour ne pas être utile”}.
\afterpage{\FloatBarrier}

%\section*{Remerciements}
%We are thankful to \textbf{...} and \textbf{....} for providing the financial support for  this project.