\section{Méthodes de Détection des Valeurs Aberrantes}

\subsection{Méthode Basée sur la Distance}

To determine whether a point is anomalous we must compare it to a set of other points. 
One natural comparison is to consider its distance from these other points.
Then the further the point in question is from the others,
the more it suggests this point is anomalous.
This works in both continuous and discrete cases whenever we are given either a distance function or a pre-computed table of pairwise distances.
The choice of which sets of points to use in this comparison distinguishes the different distance based algorithms.

Let $D \subset \mathbb{R}^n$ be an $n$-dimensional data set, 
$p$ and $q$ be data points from $D$, 
$P \subset D$ be a subset of $D$, 
and $d: D \times D \to \mathbb{R}$ be the distance between $p$ and $q$ written $d(p,q)$.
An anomaly detection algorithm must give us a function $a : D \to \mathbb{R}$ that describes how anomalous a given point is.
This induces an ordering on the points of $D$;
if $a(p) < a(q)$ for $p,q \in D$, then $p$ is less anomalous than $q$.
It is then necessary to define a threshold beyond which a point is considered anomalous.
If $\alpha \in \mathbb{R}$ is such a threshold, then any $p \in D$ where $a(p) > \alpha$ is considered an anomaly.
Such $p$ are called \textbf{absolutely anomalous}.

\subsubsection*{Similarity Measures}

A \textbf{similarity measure} is a real-valued function that describes the similarity between two objects.
A common construction is to define the similarity between two points as $\frac{1}{d(p,q)}$ for some distance function $d$.

A similarity measure can also be constructed between probability distributions.
Let $X$ and $Y$ be two $n$-dimensional random vector of (possibly) different distribution.
Recall that the covariance matrix $\Sigma$ of $X$ is an $(n \times n)$ matrix whose $(i,j)$-entry is the variance between the $i$-th and $j$-th positions of $X$.
Let $\Sigma_X$ and $\Sigma_Y$ be the covariance matrices of $X$ and $Y$ respectively.
The \textbf{Bhattacharyya distance} is defined as 

Given a collection of identically distributed samples, the covariance matrix can be estimated.

We can consider a single fixed point to be a probability distribution.
Then the Bhattacharyya distance between the point distribution and another distribution can be considered.
This special case is called the \textbf{Mahalanobis distance}.

\[
\sqrt{(p - q)^T \Sigma^{-1} (p - q)}
\]

Suppose $\Sigma$ is diagonal.
This occurs when there is no correlation i.e. linear relationship between dimensions.
This is a necessary but insufficient condition of the dimensions being independent.
In this case the Mahalanobis distance is

\[
\sqrt{\sum_{i=1}^n \frac{(p_i - q_i)^2}{\sigma_i^2}}
\]

where $\sigma_i^2$ is the standard deviation of the $i$-th dimension.
If $\Sigma$ is the identity matrix, then we recover the \textbf{Euclidean distance}

\[
\sqrt{\sum_{i=1}^n (p_i - q_i)^2}
\]

When taking the Euclidean distance a linear normalization can be first applied to each dimension so that each entry lies between $[-1,1]^n$.

% p35

\textbf{Minkowski distance}

\textbf{Jacob distance}

\textbf{Jaccard distance}

\textbf{Cosine distance}

\subsubsection*{Distance-Based Approaches}

These distance functions can be applied to create some basic anomaly detection algorithms.
These ideas will later be extended to more complex algorithms.

\textbf{Distance to All Points}

\textbf{Distance to Nearest Neighbour}

\textbf{Average Distance to $k$ Nearest Neighbours}

\textbf{Median Distance to $k$ Nearest Neighbours}

\subsection{Méthode Basée sur la Densité}
% p97

\subsubsection*{Local Outlier Factor (LOF)}

\subsubsection*{DBSCAN}

\subsection{L’écart Type}
% p8

\subsubsection*{Isolation Forest \cite{A15}}

The previously discussed approaches first construct models of what normal points look like, and then identify points that do not fit this model.
Instead anomalous points can be explicitly identified.
This allows the use of sampling techniques that increases the algorithmic speed and while decreasing memory requirements.