\phantomsection
\begin{thebibliography}{99}

\bibitem{A1} Aggarwal, C.C. [2017],  \textit{Outlier Analysis} (2nd ed.), {\it Springer.} 

\bibitem{AYU} Aggarwal, C. C. et YU, P. S. (2001). Outlier detection for high dimensional data. In ACM Sigmod Record, pages 37–46. ACM.

\bibitem{A2} Maimon, O.,  Rokach, L. [2010], \textit{Data Mining and Knowledge Discovery Handbook}, {\it Springer}. 

\bibitem{A3} Prasanta, G., et al. [2011], A Survey of Outlier Detection Methods in Network Anomaly Identification. {\it Oxford University Press, 54 (4), 570--588}

\bibitem{A4} Ranga, S.N.N.R., [2019], Outlier Detection: Techniques and Applications: A Data Mining Perspective. {\it Springer Nature Switzerland AG; 1st ed}

\bibitem{A5} Manish, G., et al., [2014], Outlier Detection for Temporal Data, IEEE Transactions on Knowledge and Data Engineering 26(9), 2250–2267.

\bibitem{A6} Kandanaarachchi, S., Hyndman, R.J. [2019], Dimension reduction for outlier detection using DOBIN, Monash Business School.

\bibitem{A7} Priyanga, D.T., et al., [2019],  Anomaly detection in high-dimensional data.

\bibitem{A8} Aggarwal, C.C., Sathe, S. [2017], Outlier Ensembles, an Introduction, {\it Springer.}

\bibitem{Aurore} Aurore Archimbaud (2018). Détection non-supervisée d’observations
atypiques en contrôle de qualité : un survol. Journal de la Société Française de Statistique, Vol. 159 No. 3 1-39


\bibitem{A9} Badr, W., \newhref{https://towardsdatascience.com/5-ways-to-detect-outliers-that-every-data-scientist-should-know-python-code-70a54335a623}{5 ways to detect outliers that every data scientist should know}, towardsdatascience.com 

\bibitem{A10} Mehrotra, K.G., Mohan, C.K., Huang, H. [2017],  Anomaly Detection Principles and algorithms, {\it Springer}. 

\bibitem{A11}  Arora, L. [2019], \newhref{https://www.analyticsvidhya.com/blog/2019/02/outlier-detection-python-pyod/}{An Awesome Tutorial to Learn Outlier Detection in Python using PyOD Library}, on Analitics Vidhya. 

\bibitem{A12} Santoyo, S. [2017], \newhref{https://towardsdatascience.com/a-brief-overview-of-outlier-detection-techniques-1e0b2c19e561}{A Brief Overview of Outlier Detection Techniques}, on towardsdatascience.com.
\bibitem{A13} He, Z., Deng, S., Xu, X. [2005], A Unified Subspace Outlier Ensemble Framework for Outlier Detection, Advances in Web Age Information Management.
\bibitem{A14} Lazarevic, A., Kumar, V. [2005], Feature Bagging for Outlier Detection, ACM KDD Conference.
\bibitem{A15} Fei, T.L., Ting, K.M., Zhou, Z.H. [2008], Isolation Forest, 2008 Eighth IEEE International Conference on Data Mining: 413–422.

\bibitem{DBSCAN} Ester, M., Kriegel, H.P., Sander, J., Xu, X. [1996], A density-based algorithm for discovering clusters in large spatial databases with noise, AAAI Press: 226--231.

\bibitem{DP_OW} Orchard, T., Woodbury, M. [1972], A Missing Information Principle: Theory and Applications, Berkeley Symposium on Mathematical Statistics and Probability, University of California Press.  
\bibitem{DP_T} Torgo, L. [2017], Data Mining with R (2nd edition), CRC Press.
\bibitem{DP_CBK} Chandola, V., Banerjee, A., Kumar, V. [2007], Outlier detection: a survey, Technical Report TR 07-017, Department of Computer Science and Engineering, University of Minnesota.
\bibitem{DP_HA} Hodge, V., Austin, J. [2004], A survey of outlier detection methodologies, Artif.Intell.Rev., 22(2):85-126.
\bibitem{DP_HPC} \newhref{https://tall.life/height-percentile-calculator-age-country/}{Height Percentile Calculator, by Age and Country}, on tall.life 

\bibitem{EIF} Hariri, S., Kind, M. C., Brunner, R. J. [2018], Extended Isolation Forest, Computing Research Repository.

\bibitem{HDBSCAN} Campello, R., Moulavi, D., Sander, J. [2013], Density-Based Clustering Based on Hierarchical Density Estimates, Advances in Knowledge Discovery and Data Mining, Springer Berlin Heidelberg: 160--172.

\bibitem{HDBSCAN_code} \newhref{https://hdbscan.readthedocs.io/en/latest/how_hdbscan_works.html}{How HDBSCAN Works} [2016] McInnes, L., Healy, J., Astels, S.

\bibitem{TS_FH} Findley, D.F., Hood, C.C., X-12-ARIMA and its Application to Some Italian Indicator Series,  U.S. Bureau of the Census.
\bibitem{TS_FMBOC} Findley, D.F., Monsell, B.C., Bell, Otto and Chen [1998], New Capabilities and Methods of the X-12-ARIMA Seasonal Adjustment Program, U.S. Bureau of the Census. 
\bibitem{TS_ICTSA} An Introductory Course on Time Series Analysis, Australian Bureau of Statistics. 
\bibitem{TS_SAETS} Seasonal Adjustment of Economic Time Series, Singapore Department of Statistics.
\bibitem{TS_JL} T.Jackson, M.Leonard, Seasonal Adjustment Using The X12 Procedure, SAS Institute.
\bibitem{W_MCC} \newhref{https://en.wikipedia.org/wiki/Matthews_correlation_coefficient}{Matthews Correlation Coefficient} (MCC) on Wikipedia.

\bibitem{PCA}
\newhref{https://en.wikipedia.org/wiki/Principal_component_analysis}{Principal Component Analysis} on Wikipedia

\bibitem{tsay} Ruey S. Tsay [1988], level shifts, and variance changes in time series.

\bibitem{chenliu} Chen, C. and Liu, Lon-Mu (1993). Joint Estimation of Model Parameters and Outlier Effects in
Time Series. Journal of the American Statistical Association, 88(421), pp. 284-297. doi: 10.2307/2290724

\bibitem{dj30} 
\newhref{https://www.kaggle.com/timoboz/stock-data-dow-jones}{EOD data for all Dow Jones stocks}

\bibitem{Zhang} 
Zhang, J., Lou, M., Ling, T. W. et Wang, H.(2004). Hos-miner : a system for detecting outlyting subspaces of high-dimensional data. In Proceedings of the Thirtieth International Conference on Very Large Data Bases, volume 30, pages 1265–1268. VLDB Endowment

\bibitem{Zi} 
Zimek, A., Kriegel, H.-P., Kröger, P., Schubert, E. (2009). Outlier detection in axis-parallel subspaces of high dimensional data. In Pacific-Asia Conference on Knowledge Discovery and Data Mining, pages 831–838. Springer.

\bibitem{LOF}
Kriegel H.-P. et al. (2000) "LOF:  Identifying  density-based  local outliers," in Proceedings  of  the  ACM  SIGMOD  International  Conference  on  Management of Data (ACM, New York), pp. 93-104

\bibitem{M1} 
Müller, E., Assent, I., Iglesias S, P., Mulle, Y. et Bohm, K. (2012). Outlier ranking via subspace analysis in multiple views of the data. In IEEE 12th International Conference on Data Mining (ICDM), pages 529-538. IEEE.

\bibitem{M2} 
Müller, E., Assent, I., Steinhausen, U. et Seidl T. (2008). OutRank : ranking outliers in high dimensional
data. In ICDEW 2008, IEEE 24th International Conference on Data Engineering Workshop, pages 600–603. IEEE

\bibitem{M3} 
Müller, E., Schiffer, M. et Seidl, T. (2010b). Adaptive outlierness for subspace outlier ranking. In Proceedings
of the 19th ACM International Conference on Information and Knowledge Management, pages 1629–1632. ACM.

\bibitem{M4} 
Müller,  E., Schiffer, M. et Seidl T. (2011). Statistical selection of relevant subspace projections for outlier
ranking. In IEEE 27th International Conference on Data Engineering (ICDE), pages 434–445. IEEE.

\bibitem{JZ}
Jian T., Zhixiang C.,
Ada W. F., David, W C., Capabilities of outlier detection schemes inlarge datasets, framework and methodologies, Springer-Verlag London Limited 2006


\bibitem{ROC}
\newhref{https://en.wikipedia.org/wiki/Receiver_operating_characteristic?oldid=375634814}{ROC curve}
\bibitem{Article_source} He, Z., Xu, X.,  Deng, S. [2005], \newhref{https://arxiv.org/pdf/cs/0507065.pdf}{A Fast Greedy Algorithm for Outlier Mining}. 

\bibitem{Hleukemia} Hastie, T., \newhref{https://web.stanford.edu/~hastie/CASI_files/DATA/leukemia.html}{Leukemia dataset}.
\bibitem{BLMMP} Leduc, O., Macfie, A., Maheshwari, A., Pelletier, M., Boily, P. [2019], Feature Selection and Dimension Reduction, \textit{Data Science Report Series}, Data Action Lab blog. 
\bibitem{LB} Leduc, O., Boily, P. [2019], \newhref{https://www.data-action-lab.com/2019/07/31/boosting-with-adaboost-and-gradient-boosting/}{Boosting with AdaBoost and Gradient Boosting}, Data Action Lab blog.
\end{thebibliography}







