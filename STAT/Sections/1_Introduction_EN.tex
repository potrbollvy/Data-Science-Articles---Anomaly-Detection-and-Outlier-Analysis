\section{Introduction}
%%%%%%%%%%%%%%%%%%%%%%%%%%%%%%%Section 1%%%%%%%%%%%%%%%%%%%%%%%%%%%%%%%%%%% 

%\label{sec:Basics.of.Stats}
In general, statistics can be divided into two categories based on their purposes: \textbf{descriptive statistics} and \textbf{inferential statistics}. \newl As its name implies, \textbf{descriptive statistics} aim to describe the collected data. Examples include:
\begin{itemize}[noitemsep]
\item sample size (overall and/or subgroups);
\item demographic breakdowns of participants;
\item measures of central tendencies (e.g., mean, median, mode, etc.);
\item measures of variability (e.g., sample variance, minimum, maximum, interquartile range, etc.);
\item measures related to higher distribution moments (skew, kurtosis, etc.);
\item non-parametric measures (minimum, maximum, various quantiles);
\item derived measures (correlation coefficients), etc.\end{itemize}
They can be presented as a single number, in a summary table, or even in graphical representations (e.g., histogram, pie chart, etc.). Descriptive statistics can be extended to summarise \textbf{multivariate} behaviours, \textit{via} sample correlations, contingency tables, scatter plots, etc. \par
Descriptive statistics not only provide an easily undertandable \textbf{overview} of the dataset; they also give analysts a chance to study the collected sample and investigate two important questions: \begin{itemize}[noitemsep]
\item is the sample compatible with our understanding of the situation?
\item is the sample representative of the underlying population?
\end{itemize}
\textbf{Inferential statistics}, on the other hand, aim to facilitate the process of inference (\textbf{induction}) to the general population from which the sample is drawn.
\newl Our review of statistical methods is by necessity quite brief; further details can be found in \cite{SA_SA,SA_KNNL,SA_HW,SA_BB,SA_SS,SA_R,SA_Re}.  

