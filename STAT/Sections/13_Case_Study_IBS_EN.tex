\section[Case Study: ANCOVA for a Clinical Study]{Case Study: Covariance Analysis of the Effect of a Probiotic Agent on IBS}\label{sec:CCNM}
We finish this report by providing a high-level summary of a project involving ANCOVA (see Section~\ref{ANCOVA}), lifted from the Executive Summary in \cite{CQADS_IBS}. \newl 
\textbf{Irritable Bowel Syndrome} (IBS) is a functional colonic disease with high prevalence. Typical symptoms include ``chronic abdominal pain, discomfort, bloating, and alteration of bowel habits'' [Wikipedia]; it has been linked to chronic pain, fatigue, and work absenteeism and is considered to have a severe impact on quality of life [Par\'e \textit{et al.} (2006), Maxion-Bergemann \textit{et al.} (2006)]. \par Although there is no known cure for IBS, there are treatments that attempt to relieve symptoms, including dietary adjustments, medication and psychological interventions.
\newl In 2010, the \textit{Canadian College of Naturopathic Medicine} (CCNM) was commissioned to conduct a study to investigate the effect of a probiotic agent on IBS. The study's details and a preliminary data analysis using \textbf{hierarchical linear models} (HLM) are not publicly available, but its key findings are that a strong placebo/expectation effect is present in the early stages of the study (which is not entirely surprising given the nature of the phenomenon under study), and that there is no strong statistical evidence to suspect that the agent itself has much of an effect on mild to moderate IBS [Herman, Cooley, Seely (2011)].\newl  The sponsor has expressed interest in determining whether these findings still hold when the trial data is examined using \textbf{analysis of covariance} (ANCOVA), a general linear model which evaluates whether the population means of a dependent/response variables (in this case, \textit{IBS Severity} or a measure of \textit{Quality of Life} (QoL)) are equal across levels of a categorical independent variable (in this case, two treatment effects over time), while statistically controlling for the effects of covariates (in this case, the baseline scores for IBSS and QoL). \par By comparison with the more traditional analysis of variance (ANOVA), ANCOVA can be used to increase the likelihood of finding a significant difference between treatment groups (when one exists) by reducing the within-group error variance.\newl While some of the results looked promising (in particular for severe IBS sufferers), no statistical evidence for treatment effect was found at the 95\% significance level; furthermore, even had evidence been found at that level, design and recruitment issues would have called their practical significance into question. \newl In 2013, CCNM conducted a second study to investigate the effect of the probiotic agent, this time focusing on severe IBS. The results are provided in the report ``Covariance Analysis of IBS Study II'' \cite{CQADS_IBS}.