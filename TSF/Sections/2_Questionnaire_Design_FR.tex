\section{Design de questionnaires}
\begin{tcolorbox}[title=Un paradoxe de la vie moderne]
Peronne n'aime \^etre recens\'e, mais donnez-moi une page de profil et je passerai toute la journée à vous dire qui je suis. \\[-0.6cm]
\begin{flushright}
-- Max Berry, \textit{Lexicon}, 2013
\end{flushright}
\end{tcolorbox}
\noindent 
Un \textbf{questionnaire} est une suite de questions visant à obtenir de l'information sur un sujet auprès d'un répondant. Les principes de conception varient en fonction du sujet et du mode de collecte des données; il reste prudent de tâter le terrain en essayant auparavant une variété de questionnaires sur une population pilote.

\subsection{Principes fondamentaus} En général, les questionnaires devraient: 
\begin{itemize}[noitemsep] 
\item être aussi bref que possible, sans questions inutiles;
\item être accompagnés d'instructions claires et concises;
\item garder les intérêts de la personne interrogée en tête;
\item mettre l'accent sur la confidentialité;
\item garder un ton sérieux et courtois;
\item être exempts d'erreurs et présentés de manière attrayante;
\item être formulés de façon claire et précise; 
\item être conçus de manière à ce qu'on puisse y répondre avec précision, et 
\item ordonnés avec soin. 
\end{itemize}
\subsection{Types de questions} L'unité de base du questionnaire est, bien entendu, la \textbf{question}, qui se présente sous deux formes: 
\begin{itemize}[noitemsep]
\item la question \textbf{fermée}, avec un nombre fixe de choix de réponses prédéterminés, mutuellement exclusifs, et collectivement exhaustifs (et qui devrait toujours inclure une catégorie ``Autre (veuillez préciser)'’ afin de contrecarrer la perte d'expressivité), et
\item la question \textbf{ouverte}, qui sert entre autres à identifier les choix de réponses communs à utiliser dans les questions fermées d’un questionnaire ultérieur. 
\end{itemize}
\subsection{Considérations relatives à la formulation} Il est bien connu que la formulation des questions peut influencer les réponses d'un questionnaire \cite{DC_G}; il est bon de garder les \textbf{considérations de formulation} suivantes en tête lors de l'élaboration de questionnaires:   
\begin{itemize}[noitemsep] 
\item éviter les \textbf{abréviations} et \textbf{jargon} (``Votre organisation utilise-t-elle des pratiques TTWQ?'');
\item éviter d'utiliser des \textbf{termes complexes} quand des termes plus simples font l'affaire (``Combien de fois avez-vous été défenestré?'' vs ``Combien de fois vous a-t-on jeté par la fenêtre?'');
\item précisez le \textbf{cadre de référence} (``Quel est votre revenu annuel?'' vs ``Quel était le revenu total de votre ménage, toutes sources confondues, avant impôts et déductions, en 2017?’’);
\item rendre la question aussi \textbf{précise} que possible ("Combien de carburant votre compagnie de déménagement a-t-elle utilisé l'an dernier?" vs ``Combien votre compagnie de déménagement a-t-elle dépensée en carburant l’an dernier?''');
%\item veiller à ce que toutes les personnes interrogées puissent répondre aux questions;
\item éviter les questions à \textbf{double volet} (``Prévoyez-vous laisser votre voiture à la maison et prendre le train léger afin de vous rendre au travail au cours de l'année à venir?" vs ``Prévoyez-vous laisser votre voiture à la maison au cours de l'année à venir? Si oui, prévoyez-vous prendre le train léger afin de vous rendre au travail?"), et 
\item éviter les \textbf{questions tendancieuses} (consulter le toujours excellent \textit{Yes, Prime Minister} \cite{DC_YPM} pour un exemple qui n’est pas si facétieux que ça, en fin de compte).
\end{itemize}
\subsection{Ordre des questions} L'ordre dans lequel les questions sont présentées est tout aussi important que leur formulation. Les questionnaires doivent être conçus de manière à \textbf{dérouler sans heurts} et à \textbf{suivre une suite logique} (c'est-à-dire logique pour la personne interrogée):  
\begin{enumerate}[noitemsep]
\item commencer avec une \textbf{introduction} qui fournit le titre, le sujet et l'objectif de l'enquête;
\item demander la \textbf{coopération} du répondant et expliquer l'importance de l'enquête et la manière dont les résultats seront utilisés;
\item indiquer le degré de \textbf{confidentialité} et fournir une date limite et une adresse de contact;
\item commencer par une série de questions \textbf{faciles} et \textbf{intéressantes} afin d'établir la confiance du répondant;
\item grouper les questions semblables sous une \textbf{même rubrique};
\item n'introduire les \textbf{sujets sensibles} que lorsque un rapport de confiance avec le répondant est susceptibles de s'être développé;
\item laisser un peu d'espace et/ou de temps pour les \textbf{commentaires supplémentaires}, et 
\item \textbf{remercier} le répondant de sa participation.
\end{enumerate}
De nombreux ouvrages discutent du design des questionnaires (voir \cite{DC_O}, par exemple). Il est facile de passer beaucoup trop de temps à concevoir le questionnaire ``parfait''; il est bon de se rappeller que sans plan d'échantillonnage solide, les données recueillies, quelles qu'elles soient, peuvent être de telle piètre qualité qu’il devient impossible de tirer des conclusions exploitables. 