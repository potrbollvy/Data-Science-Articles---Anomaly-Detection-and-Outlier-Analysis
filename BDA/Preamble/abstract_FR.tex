\Abstract{L'analyse bayésienne est parfois dénigrée par les analystes de données, en partie à cause de l'élément d'arbitraire perçu associé à la sélection d'une loi a priori significative pour un problème spécifique et des (précédentes) difficultés liées à la production de la loi a posteriori dans toutes les situations sauf les plus simples. D'autre part, nous avons entendu dire que ``si les analystes de données classiques ont besoin d'un grand nombre d'astuces intelligentes pour exploiter leurs données, les Bayésiens n'ont jamais vraiment besoin que d'une seule r\`egle.'' Avec l'avènement des méthodes d'échantillonnage numérique efficace, les analystes de données modernes ne peuvent pas hésiter à ajouter la flèche bayésienne à leur carquois. Dans ce bref rapport, nous présentons les concepts de base qui sous-tendent l'analyse bayésienne, ainsi qu'un petit nombre d'exemples bien connus qui illustrent les points forts de l'approche.}