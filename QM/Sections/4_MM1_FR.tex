%%%%%%%%%%%%%%%%%%%%%% section 4%%%%%%%%%%%%%%%%%%%%%
\section{Le système $M/M/1$}
Nous allons maintenant discuter du système de file d'attente non trivial le plus simple.  
\subsection{Principes fondamentaux} Un système de file d'attente $M/M/1/\textrm{GD}/\infty/\infty$  dispose d'intervalles d'arrivée et de temps de service dont les distributions respectives sont exponentielles, et un serveur unique. On peut le modéliser à l'aide d'un processus de naissance et de mort où
\begin{align*}
\lambda_{j} &= \lambda, \ j=0,1,2,\ldots \\ 
\mu_{0} &= 0 \\
\mu_{j} &= \mu,\  j=1,2,3,\ldots 
\end{align*}
En substituant ces taux de natalité et de mortalité dans (\ref{eq:ssbr}), on obtient  
$$\pi_{j} = \frac{\lambda^{j} \pi_{0}}{\mu^{j}}=\rho^j \pi_0,$$ où $\rho = \lambda/\mu$ représente l'\textbf{intensité du trafic} dans le système. \newpage\noindent 
Étant donné que le système doit se trouver exactement dans l'un des états à tout moment, la somme de toutes les probabilités se doit d'\^etre $1$: $$\pi_{0}+\pi_1 + \pi_2+\cdots = \pi_0(1+\rho+\rho^{2}+\cdots ) = 1.$$
Lorsque $0 \leq \rho < 1$, la série converge vers $\frac{1}{1-\rho}$, d'où l'on dérive $$\pi_{0}\cdot \frac{1}{1-\rho} = 1 \implies \pi_0 = 1-\rho \implies \pi_{j} = \rho^{j} \pi_0 = \rho^j (1-\rho);$$
c'est la \textbf{probabilité de retrouver le système  dans le  $j^{\text{e}}$ état dans le régime stable}. Si au contraire $\rho \geq 1$, la série diverge et le système n'atteint pas de régime stable. Intuitivement, cela se produit lorsque $\lambda \geq \mu$: lorsque le taux d'arrivée est supérieur au taux de service, l'état du système croît sans cesse et la file d'attente ne se dégage jamais.
\newl À partir de maintenant, nous allons présumer que $\rho < 1$ afin de garantir l'existence des probabilités  $\pi_{j}$ dans le régime stable, à partir desquelles nous pouvons déterminer plusieurs quantités d'intérêt. \par En supposant que l'état d'équilibre a été atteint, on peut montrer que $L$, $L_{s}$, et $L_{q}$ sont, respectivement:
\begin{align*}
L &= \frac{\lambda}{\mu - \lambda}=\frac{\rho}{1-\rho}\\
L_{s} &= \rho\\
 L_{q} &= \frac{\rho^{2}}{1-\rho}.
 \end{align*}
à l'aide de la loi de Little, nous pouvons également obtenir $W$, $W_{s}$, et $W_{q}$ en divisant chacune des valeurs correspondantes de $L$ par~$\lambda$:
\begin{align*}
W &= \frac{1}{\mu - \lambda}\\
W_{s} &= \frac{1}{\mu}\\
W_{q} &= \frac{\lambda}{\mu(\mu-\lambda)}.
 \end{align*}
On note comme prévu que $W,W_q\to +\infty$ quand $\rho\to 1$. En revanche, $W_{q}\to 0$ et $W\to \frac{1}{\mu}$ (le \textbf{temps moyen de service}) lorsque $\rho\to 0$.
\begin{Exemple} (selon \cite{QS_W}) En moyenne, dix clients arrivent à un guichet à toutes les heures. Si le client moyen est servi en 4 minutes, et si les intervalles d'arrivée et le temps de service suivent tous deux des distributions exponentielles, alors: \begin{itemize}[noitemsep]
	\item[(a)] Quelle est la probabilité que le guichet se retrouve au repos? 
	\item[(b)] Sans compter le client qui se fait servir, combien de clients font la queue au guichet, en moyenne? \item[(c)] Combien de temps, en moyenne, un client passe-t-il dans le système de file d'attente (y compris le temps de service)?
	\item[(d)] En moyenne, combien de clients seront servis par le caissier à chaque heure?
\end{itemize}
\textbf{Solution:} nous faisons affaire à un système de file d'attente $$M/M/1/\textrm{GD}/\infty/\infty$$ pour lequel $\lambda = 10$ clients/h et $\mu = 15$ clients/h, d'où   $\rho = 10/15 = 2/3$.
\begin{itemize}[noitemsep]
	\item[(a)] Puisque  $\pi_{0} = 1 - \rho = 1/3$, le guichet se retrouve au repos $1/3$ du temps.  
	\item[(b)] Il y a $L_{q} = \rho^{2}/(1-\rho) = 4/3$ clients en ligne pour le guichet, en moyenne. 
	\item[(c)] On sait que $L = \lambda/(\mu - \lambda) = 10/(15-10) = 2$,, d'où  $W = L/\lambda = 0.2 \textrm{ h} = 12 \textrm{ min}$.
	\item[(d)] Si le guichet est toujours occupé, on sert en moyenne $\mu=15$ clients par heure. Selon (a), nous savons que la caissière n'est occupée que les deux tiers du temps, c'est-à-dire qu'à chaque heure, elle ne sert en moyenne que $15  \cdot 2/3 = 10 $ clients. C'est un résultat raisonnable car, dans le régime stable, $10$ clients entrent dans le système  et $10$ clients quittent le système à chaque heure.
\end{itemize}
\end{Exemple}
\begin{Exemple} (selon \cite{QS_E}) Supposons que tous les propriétaires de voitures fassent le plein lorsque leur réservoir est exactement à moitié vide. Supposons également que $7.5$ clients arrivent en moyenne à chaque heure à une station-service qui n'a qu'une seule pompe et qu'il faut en moyenne à chaque voiture $4$ minutes pour faire le plein. Supposons finalement que les intervalles d'arrivées et les temps de service suivent tous deux des distributions exponentielles. \begin{itemize}[noitemsep]
	\item[(a)] Quelles valeurs prennent $L$ et $W$ dans ce scénario? 
	\item[(b)] Supposons que lorsqu'il y a pénurie de gaz, les achats d'essence se fassent dans la panique. On modélise ce phénomène en imaginant que tous les propriétaires de voitures achètent désormais de l'essence lorsque leur réservoir est rempli aux trois quarts exactement. Comme chaque propriétaire de voiture met désormais moins d'essence dans le réservoir à chaque visite à la station, nous supposons que la durée moyenne du temps de service  a été réduite à $10/3$ minutes. De quelle fa\c{c}on est-ce que cela affecte les valeurs de $L$ et $W$?
\end{itemize}
\textbf{Solution:} on prend pour acquis que le syst\`me de file d'attente prend la forme $$M/M/1/\textrm{GD}/\infty/\infty,$$ où $\lambda = 7.5$ voitures/h et $\mu = 60/4 = 15$ voitures/h.  Nous avons alors $\rho = 7.5/15 = 1/2$.
\begin{enumerate}[noitemsep]
	\item[(a)] Par définition, $L = \lambda/(\mu - \lambda) = 7.5/(15-7.5) = 1$ et $W = 1/7.5 \approx 0,13$ h $=7.8$ min. Dans cette situation, tout est sous contrôle et de longues files d'attente semblent peu probables.
\item[(b)] Pour le scénario où les achats se font dans la panique,   $\lambda = 2(7.5)=15$ voiture/h puisque les propriétaires de voiture font le plein 2 fois plus souvent qu'au préalable, et $\mu = 60 \cdot 3 /10  = 18$ voitures/h, d'où $\rho = \lambda/\mu = 5/6$. Dans ce cas,   
$$ L = \frac{\rho}{1-\rho} = 5 \text{ voitures, et } W = \frac{L}{\lambda} = \frac{5}{15} = 20 \text{ min}.$$
Ainsi, les achats dans la panique ont pour effet de plus que doubler le temps d'attente dans la file d'attente.
\end{enumerate}
\end{Exemple}
\noindent Dans un système $M/M/1$, on a obligatoirement $$L=\frac{\rho}{1-\rho}=-1+\frac{1}{1-\rho},$$ et on constate aisément que  $L\to\infty$ comme $\rho\to 1$. La multiplication par 5 de la valeur de $L$ lorsque $\rho$ passe de $1/2$ à $5/6$ (avec des sauts correspondants en $W$) illustre ce fait. 
\subsection{Capacité limite}
En réalité, la capacité d'une file d'attente ne saurait \^etre infinie -- elle est limitée par les exigences de l'{espace} et/ou du {temps}, ou encore par la politique d'exploitation des services. Un modèle qui tient compte des ces aspects est du ressort des \textbf{files d'attente finies}. \par Ces modèles limitent le nombre de clients autorisés dans le système de service, soit $N$. Si le système est à \textbf{capacité}, l'arrivée d'un $(N+1)^{\textrm{e}}$ client entraîne l'impossibilité d'entrer dans la file d'attente -- l'accès au système est bloqué pour ce client, qui doit quitter sans recevoir de service. \newl Les files d'attente finies peuvent également être modélisées par un processus de naissance-mort, mais avec une légère modification de ses paramètres:  \begin{align*}
\lambda_{j} &= \lambda, \ j=0,1,2,\ldots,N-1 \\ 
\lambda_{N} &= 0,\ \mu_{0} = 0 \\
\mu_{j} &= \mu,\  j=1,2,3,\ldots, N 
\end{align*}
La restriction $\lambda_{N} = 0$ distingue ce modèle de $M/M/1/\infty$. Elle rend impossible l'accès à un état supérieur à $N$. En conséquence, les modèles de file d'attente finie ont toujours un régime stable puisque même si $\lambda \geq \mu$, il ne peut jamais y avoir plus de $N$ clients dans le système.
\newl Du c\^oté mathématique, on remplace la série infinie reliant les $\pi_j$ par une série géométrique finie, qui converge quelque soit la valeur de $\rho$: 
$$ \pi_{0}+\pi_1 + +\cdots + \pi_N = \pi_0(1+\rho+\cdots +\rho^{N}) = 1,$$ \noindent d'où \begin{align*}\pi_{0}&\cdot \frac{1-\rho^{N+1}}{1-\rho} = 1 \\ &\implies \pi_0 = \frac{1-\rho}{1-\rho^{N+1}} \\ &\implies \pi_{j} = \begin{cases}\rho^{j} \pi_0 & \text{lorsque $j=0,\ldots,N$} \\ 0 & \text{lorsque $j>N$}\end{cases}\end{align*}
Puisque $L = \sum^{N}_{j=0} j \pi_{j}$ (est-ce évident?), 
$$L = \frac{\rho [1+ N \rho^{N+1} - (N+1) \rho^{N} ]}{(1-\rho)\left(1-\rho^{N+1}\right)} $$
lorsque $\lambda\neq\mu$. Comme c'est le cas dans une file d'attente  $M/M/1/\infty$, $L_{s} = 1 - \pi_{0}$ et $L_{q} = L - L_{s}$.  \newl Dans un modèle à capacité finie, seulement $\lambda - \lambda \pi_{N} = \lambda\left(1-\pi_{N}\right)$ arrivées par unité de temps entrent effectivement dans le système, en moyenne (il y a en fait $\lambda$ arrivée, mais pour $\lambda\pi_N$ d'entre elles, le système est rempli). On a alors 
$$ W = \frac{L}{\lambda\left(1-\pi_{N}\right)}\quad \text{et}\quad W_{q} = \frac{L_{q}}{\lambda\left(1-\pi_{N}\right)}.$$
A quoi cela ressemble-t-il, concrètement? 
\begin{Exemple} Imaginons un salon de coiffure, tenu par un seul barbier, contenant un total de 10 sièges. Supposons de plus, comme nous l'avons toujours fait jusqu'à présent (quoique cela ne soit pas nécessaire), que les intervalles d'arrivée sont réparties de manière exponentielle, avec une moyenne de 20 clients potentiels arrivant chaque heure au salon. Les clients qui trouvent le magasin plein n'y entrent pas (peut-être n'aiment-ils pas rester debout). Le barbier prend en moyenne 12 minutes pour couper les cheveux de chaque client (supposons que les temps de coupe soient également répartis de manière exponentielle).
\begin{itemize}[noitemsep]
	\item[(a)] En moyenne, combien de coupes de cheveux le barbier réalise-t-il par heure?
	\item[(b)] En moyenne, combien de temps un client qui entre dans le magasin passe-t-il dans le salon?
\end{itemize}
\textbf{Solution:} c'est plus simple que cela ne le paraît. Allons-y!
\begin{itemize}[noitemsep]
	\item[(a)] Le salon sera plein pour une fraction $\pi_{10}$ de toutes les arrivées. Ainsi, $\lambda\left(1-\pi_{10}\right)$ clients sont introduits dans la file d'attente à toutes les heures, en moyenne.  Puisque tous les clients qui entrent dans la file reçoivent une coupe de cheveux, le barbier donne en  moyenne $\lambda\left(1-\pi_{10}\right)$ coupes de cheveux par heure. Dans ce scénario, $N=10$, $\lambda=20$ clients/h, et $\mu=60/12 = 5 $ clients/h. Ainsi, $\rho= 20/5 = 4$ et nous avons 
	\begin{align*} \pi_{0} &= \frac{1-\rho}{1-\rho^{N+1}} = \frac{1-4}{1-4^{11}}\approx 7.15\times 10^{-7}  \text{ et} \\ \pi_{10} &= 4^{10} \pi_{0} = \frac{3}{4}.\end{align*}
Il y a ainsi $20 (1 - 3/4) = 5$ clients qui reçoivent une coupe de cheveux par heure, en moyenne. Cela signifie que l'accès à la file d'attente sera bloqué pour $20 - 5 = 15 $  clients potentiels à chaque heure, en moyenne. 	
\item[(b)] On d\'termine $W$ en calculant tout d'abord
$$L = \frac{4 [1+ (10) 4^{11} - (11) 4^{10} ]}{(1-4)\left(1-4^{11}\right)} = 9.67.$$ 
à l'aide de la formule donnée précedemment, nous obtenons 
$$ W = \frac{L}{\lambda\left(1-\pi_{10}\right)} = \frac{9.67}{5} = 1.93 \text{ h}.$$
Le salon de coiffure est bondé - le barbier serait bien avisé d'engager au moins un autre barbier !
\end{itemize}
\end{Exemple}
\noindent Quel effet l'embauche d'un deuxième coiffeur aurait-elle sur le syst\`me de file d'attente? Pour répondre à cette question, nous devons étudier les systèmes de files d'attente $M/M/c$.   
%%%%%%%%%%%_________Section 5____________%%%%%%%%%%%%%%%%%%%%
