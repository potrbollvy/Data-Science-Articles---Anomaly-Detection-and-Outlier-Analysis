\section{Introduction} La \textbf{théorie des files d'attente} est une branche des mathématiques qui étudie et modélise le comportement des files d'attente. L'article fondateur sur la théorie des files d'attente \cite{QS_Erlang} a été publié en 1909 par le mathématicien danois A.K. Erlang; il y étudiait \begin{quote} le problème de déterminer le nombre de circuits téléphoniques nécessaires afin de fournir un service qui empêcherait les clients d'attendre trop longtemps avant qu'un circuit se libère. En élaborant une solution à ce problème, il a  réalisé que le problème de la minimisation du temps d'attente était applicable à de nombreux domaines, et a commencé à développer davantage sa théorie. Le \textbf{problème du tableau téléphonique} d'Erlang a ouvert la voie à la théorie moderne des files d'attente \cite{QS_Berry}.\end{quote}
On cherche \`a répondre à des questions telles que:
\begin{itemize}[noitemsep]
\item Est-il probable que des objets/unités/personnes fassent la queue et attendent en file?
\item Quelle sera la taille de la file d'attente?
\item Combien de temps faudra-t-il attendre?
\item Quel sera le niveau d'occupation du système?
\item Quelle capacité est requise pour répondre au niveau de demande attendu?
\end{itemize}
C'est en réfléchissant à ce genre de questions que les analystes et les parties prenantes pourront anticiper les \textbf{blo\-ca\-ges} (``bottlenecks''). On pourra alors mettre en place des systèmes et des équipes plus efficaces et plus flexibles, plus performants et moins dispendieux et, en fin de compte, offrant un meilleur service aux clients et aux utilisateurs. \par La théorie des files d'attente permet également de traiter les blocages (et leur effet sur la performance du syst\`eme) de mani\`ere quantitative. On sera en mesure de donner une r\'eponse \`a une question telle que ``combien de temps devra-t-on attendre, en moyenne'', tout comme \`a une multitude d'autres questions concernant la variabilité des temps d'attente, leur distribution, la probabilité qu'un client reçoive un service m\'ediocre, extrêmement médiocre, etc \cite{QS_K1}.
\newl
Prenons un exemple simple. Supposons qu'il y ait dans une épicerie une seule ligne de caisse et un seul caissier. Si, en moyenne, un client arrive à la caisse pour payer son épicerie toutes les 5 minutes et si le scannage, l'emballage et le paiement prennent 4.5 minutes en moyenne, est-ce que l'on s'attendrait à ce que les clients doivent faire la queue ? Lorsque le problème est présenté de cette manière, notre intuition nous dit qu'il ne devrait pas y avoir de file d'attente et que le caissier devrait rester inactif, en moyenne, 30 secondes toutes les 5 minutes, n'étant occupé que 90\% du temps. Personne n'aura alors besoin d'attendre avant d'être servi ! \par Si vous avez déjà fréquenté une épicerie, vous savez que ce n'est pas ce qui se passe dans la réalité; beaucoup d'acheteurs font la queue, et ils doivent attendre longtemps avant d'être servis. 
 Fondamentalement, le phénomène de \textbf{files d'attente} se produit pour trois raisons :
  \begin{itemize}
 \item les \textbf{arrivées irrégulières} -- les clients n'arrivent pas à la caisse selon un horaire régulier; ils le font parfois éloignés et parfois rapprochés les uns des autres, de sorte \`a qu'ils y a chevauchement (qui entraînent automatiquement des files d'attente);
 \item les \textbf{tâches de taille irrégulière} -- les achats ne sont pas tous traités en 4.5 minutes; un client qui fait les courses pour une famille nombreuse aura besoin de beaucoup plus de temps qu'une personne qui ne fait les courses que pour elle-même, par exemple (lorsque cela se produit, le chevauchement est à nouveau un problème car de nouveaux clients  arriveront pour payer leurs courses pendant que les clients actuels sont encore en train de se faire tra\^{\i}ter), et  
\item le \textbf{gaspillage} -- le temps perdu ne peut jamais être rattrapé; les clients se chevauchent parce que le deuxième client est arrivé trop tôt, avant que le premier n'ait eu le temps de finir de se faire tra\^{\i}ter; mais ce n'est peut-être pas la faute du deuxième client; peut-être le premier client serait arriv\'e \`a la caisse plus tôt, mais il a perdu du temps à lire un magazine à potins pendant que le caissier était inactif! Ils ont manqué leur chance d'être servis rapidement et, par conséquent, ont fait attendre le deuxième client.
\end{itemize}
Les arrivées irrégulières et les t\^aches de taille irr\'eguli\`ere provoquent automatiquement des files d'attente. La seule mani\`ere de ne pas avoir de file d'attente est de s'assurer que les tâches soient uniformes, que les arrivées soient régulières et que le caissier ait exactement assez de travail pour faire face aux arrivées. Même lorsque le caissier est à peine occupé, les arrivées irrégulières ou les arrivées \textbf{en rafales} peuvent provoquer de l'attente. \par En général, les files d'attente \textbf{s'aggravent} lorsque les conditions suivantes sont présentes:
\begin{itemize}
\item une \textbf{utilisation élevée des serveurs} -- plus la caissière est occupée, plus il lui faut de temps pour se remettre du temps perdu; 
\item une \textbf{variabilit\'e \'elev\'ee} -- plus la variabilité des arrivées ou de la taille des tâches est importante, plus il y a de gaspillage et de chevauchement (files d'attente), et 
\item un \textbf{nombre insuffisant de serveurs} -- Moins de caissiers signifie moins de capacité à absorber les rafales \`a l'arrivée, ce qui entraîne une plus grande perte de temps et une plus forte utilisation.
\end{itemize}