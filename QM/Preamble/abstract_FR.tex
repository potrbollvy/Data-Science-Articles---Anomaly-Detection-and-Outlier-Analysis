%----------------------------------------------------
%	ABSTRACT
%----------------------------------------------------

\Abstract{La \textbf{théorie des files d'attente} est une branche des mathématiques qui étudie et modélise le comportement des files d'attente. En tant que volet de la recherche opérationnelle, elle combine des éléments de diverses disciplines quantitatives, mais elle ne fait que rarement partie de la boîte à outils de l'analyste de données. \par \bigskip Dans ce rapport, nous présentons la terminologie et le contexte de base des modèles de files d'attente (y compris la notation de Kendall-Lee, les processus de naissance-mort, et la formule de Little), ainsi que le modèle de file d'attente le plus couramment utilisé : $M/M/c$. Nous décrivons également une application au contrôle de pré-embarquement dans les aéroports canadiens. \par\smallskip \ 
}
%----------------------------------------------------