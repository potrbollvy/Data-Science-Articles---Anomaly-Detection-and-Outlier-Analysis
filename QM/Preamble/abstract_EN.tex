%----------------------------------------------------
%	ABSTRACT
%----------------------------------------------------

\Abstract{\textbf{Queuing theory} is a branch of mathematics that studies and models the act of waiting in lines, or queues. As a topic in operational research, it combines elements of a variety of quantitative disciplines, but it is not often part of the data analyst's toolbox. \par \bigskip In this report, we introduce the terminology and basic framework of queueing models (including Kendall-Lee notation, birth-death processes, and Little's formula), as well as the most commonly-used queueing system: $M/M/c$. We also describe an application to pre-board screening at Canadian airports. \par\smallskip \
}

%----------------------------------------------------