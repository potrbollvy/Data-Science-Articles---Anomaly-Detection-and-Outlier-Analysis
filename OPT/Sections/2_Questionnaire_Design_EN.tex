\section{Questionnaire Design}
\begin{tcolorbox}[title=A Modern Paradox]
People resist a census, but give them a profile page and they'll spend all day telling you who they are. \\[-0.6cm]
\begin{flushright}
-- Max Berry, \textit{Lexicon}, 2013
\end{flushright}
\end{tcolorbox}
\noindent 
A \textbf{questionnaire} is a sequence of questions designed to obtain information on a subject from a respondent. Design principles vary according to the subject matter and the mode of data collection, but we strongly encourage pre-testing a variety of questionnaires.\subsection{Questionnaire Design Basics} In general, questionnaires should: 
\begin{itemize}[noitemsep] 
\item be as brief as possible, with no wasted questions;
\item be accompanied by clear and concise instructions;
\item keep the respondent's interests in mind;
\item emphasise confidentiality;
\item be serious and courteous in tone;
\item be free of mistakes and laid out attractively;
\item be worded clearly; 
\item be designed to be accurately answered, and 
\item be ordered attentively. 
\end{itemize}
\subsection{Question Types} The basic questionnaire unit is, of course, the \textbf{question}, which comes in two flavours: 
\begin{itemize}[noitemsep]
\item \textbf{closed}, with a fixed number of pre-determined mutually exclusive and collectively exhaustive answer choices (and should always include an ``Other (Please Specify)'' category to counteract the loss of expressiveness of such questions), and
\item \textbf{open}, which serves to identify common response choices to be used as closed question choices in subsequent questionnaires. 
\end{itemize}
\subsection{Wording Considerations} It is well known that the wording of the questions can influence a questionnaire's responses \cite{DC_G}; please keep the following \textbf{wording considerations} in mind when designing a questionnaire:   
\begin{itemize}[noitemsep] 
\item avoid \textbf{abbreviations} and \textbf{jargon} (``Does your organization use any TTWQ practices?'');
\item do not use words and terminology that are \textbf{too complex} (``How often have you been defenestrated?'' vs.\@ ``How often have you been thrown out of a window?'');
\item specify the \textbf{frame of reference} (``What is your income?'' vs.\@ ``What was your household's total income form all sources before taxes and deductions in 2017?'');
\item make the question as \textbf{specific} as possible (``How much fuel did your moving company use during the last year?'' vs.\@ ``How much did your moving company  spend on fuel during the last year?'');
\item ensure that the questions can be answered by all respondents;
\item avoid \textbf{double-barrelled} questions (``Do you plan to leave your car at home and take the bus to work during the coming year?'' vs. ``Do you plan to leave your car at home in the coming year? If so, do you plan to take the bus to work?''), and 
\item avoid leading questions (see the always excellent \cite{DC_YPM} for a not-so-facetious example).
\end{itemize}
\subsection{Question Order} The order of the questions is just as important as the wording. Questionnaires should be designed to \textbf{flow smoothly} and \textbf{follow a logical sequence} (logical to the respondent, that is):  
\begin{enumerate}[noitemsep]
\item start with an \textbf{introduction} which provides the title, subject, and purpose of the survey;
\item request \textbf{cooperation} and explain the importance of the survey and how the results will be used;
\item indicate the degree of \textbf{confidentiality} and provide a deadline and a contact address;
\item open with a series of \textbf{easy} and \textbf{interesting questions} to establish the respondent's confidence;
\item group similar questions under a \textbf{common heading};
\item introduce \textbf{sensitive topics} once trust and confidence are likely to have developed;
\item allow some space and/or time for \textbf{additional comments}, and 
\item \textbf{thank} the respondent for their participation.
\end{enumerate}
A lot more has been written about questionnaire design (see \cite{DC_O}, for instance). It can be surprisingly easy to get lost in the jungle and spend way too much time on the ``perfect'' design; remember that without a sound sampling plan, whatever data is collected may not prove up to the task of drawing the actionable insights that the client is really interested in seeing  answered. 